\chapter{绪论}
\pagenumbering{arabic}
\section{研究背景及意义}
{在}
{1980年,Martin Hellman\cite{hellman}提出了一个“时间空间折中”的密码分析算法,使用了预先计算好并保存在内存和磁盘里面的数据,减少了密码分析需要的时间。这个算法在1982年被Rivest提出改进,减少了密码分析过程中所需要的存储空间。

2003年7月瑞士洛桑联邦技术学院的Philippe Oechslin公布了一些实验数据,他及其所属的安全及密码学实验室(LASEC)采用了时间空间折中算法,使得密码破解的效率大大提高。他们开发的Ophcrack项目可以将一个操作系统的用户登录密码破解速度由1分41秒,提升到13.6秒\cite{PO}。该项目提供了一个破解视窗作业系统下的LAN Manager散列(比如hash文件)的程序,作者免费提供了一些Rainbow table,可以在短至几秒内破解最多14个英文字母的密码,有99.9%的成功率。从2.3版开始可以破解 NT 散列,这功能对已经关闭 LAN Manager 散列的系统(Windows Vista 的预订设定)或是长于14个字母的密码特别有用。

同年project-rainbowcrack项目开始立项,该项目基于Philippe Oechslin提出的彩虹表,用C++基本实现了对MD5、SHA-1算法的低位数低密钥空间的破解[3]。接着出现了一个分布式彩虹表项目Free Rainbow Tables[4],这个项目的分布式系统是基于伯克利开放式网络计算平台(BOINC)。


}
\section{国内外研究现状与进展}
\section{本文研究现状}
\section{论文组织结构}
