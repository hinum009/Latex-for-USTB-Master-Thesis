\appendix
\chapter{MATLAB程序}
\section{破解成功率}
\begin{lstlisting}
\% 1 - (1 - 1 / N)^(m(1) + m(2) + m(3) + ... + m(t - 1))
\% m(1) = m, m(i) = N * (1 - (1 - 1 / N) ^ m(i - 1))
function ret = calc_success_probability(n, min, max, t, m, table_count)
format long g;
arr = zeros(1, t - 1);
arr(1) = m;
format long g;
N = calc_N(n,min,max);
for i = 2 : t - 1
    format long g;
	arr(i) = N * (1 - (1 - 1 / N) ^ arr(i - 1));
end;
exp = 0;
for i = 1 : t - 1
    format long g;
	exp = exp + arr(i);
end;
format long g;
success_probability_1 = 1 - (1 - 1 / N) ^ exp;
success_probability_table_count=1-(1-success_probability_1)^table_count;
ret = success_probability_table_count;

function ret = calc_N( n, min, max)
arr = zeros(1, max - 1);
for i = 1 : max
	arr(i) = n^i;
end;
exp = 0;
for i = 1 : max
	exp = exp + arr(i);
end;
ret = exp;
\end{lstlisting}
\section{磁盘空间与成功率}
\begin{lstlisting}
function advantage_of_multiple_table(N, t, disk_usage_min, disk_usage_max)
grid_count = 100;
disk_usage = linspace(disk_usage_min, disk_usage_max, grid_count);
m = disk_usage * 1024 * 1024 / 16;
success_probability = zeros(1, grid_count);

for table_count = 1 : 5
  for i = 1 : grid_count
    success_probability(i)=calc_success_probability(N,t,m(i)/table_count);
    success_probability(i)=1-(1-success_probability(i))^table_count;
  end;
  if table_count ==1
    plot(disk_usage, success_probability,'-c*');
  end
  if table_count ==2
    plot(disk_usage,success_probability,'-x')
   end
  if table_count == 3
    plot(disk_usage,success_probability,'k-.')
  end
  if table_count == 4
    plot(disk_usage,success_probability,'-+')
  end
  if table_count == 5
    plot(disk_usage,success_probability,'y-o')
  end
  if N == 8353082582 & t == 2100 & table_count == 5
    plot(8000000*16*5/1024/1024, 0.999,'s','MarkerFaceColor','b',
                          'MarkerEdgeColor','b','MarkerSize',3);
    text(8000000*16*5/1024/1024, 0.999,'G','HorizontalAlignment','right',
                                           'VerticalAlignment','bottom');
  end;
  hold on;
end;
legend_int = zeros(5 - 1 + 1, 1);
for table_count = 1 : 5
	legend_int(table_count - 1 + 1, 1) = table_count;
end;
legend_str = int2str(legend_int);
legend(legend_str);
for i = 1 : grid_count
	success_probability(i) = 0.999;
end;
plot(disk_usage, success_probability, 'r');

for i = 1 : grid_count
	success_probability(i) = 1;
end;
plot(disk_usage, success_probability, 'r');

xlabel('磁盘空间(MB)');
ylabel('成功概率');
\end{lstlisting}
\section{密钥空间}
\begin{lstlisting}
function ret = calc_N( n, min, max)
arr = zeros(1, max );
for i = 1 : max
	arr(i) = n^i;
end;

exp = 0;
for i = 1 : max
	exp = exp + arr(i);
end;
format long g;
ret = exp;
\end{lstlisting}
\section{彩虹链长度}
\begin{lstlisting}
function ret = calc_t(N, success_probability, m, table_count)
success_probability_each=1-exp(log(1-success_probability)/table_count);
exp_min = log(1 - success_probability_each) / log(1 - 1 / N);

exp_all = 0;
t = 0;
next = m;

while exp_all < exp_min
	exp_all = exp_all + next;
	t = t + 1;
	next = N * (1 - (1 - 1 / N) ^ next);

	if t > 131072	% too large
		break;
	end;
end;

ret = t;
\end{lstlisting}
\section{磁盘占用空间}
\begin{lstlisting}
function ret = calc_disk_usage(m, table_count)
ret = ceil(m*16*table_count / 1024 / 1024 );
\end{lstlisting}
\section{最大破解时间}
\begin{lstlisting}
function ret = calc_max_cryptanalysis_time(t, table_count, step_speed)
ret = t*t/2 / step_speed * table_count;
\end{lstlisting}
\section{实际破解时间}
\begin{lstlisting}
                               rate_of_each_table * 1 
(1 - rate_of_each_table) ^ 1 * rate_of_each_table * 2 
(1 - rate_of_each_table) ^ 2 * rate_of_each_table * 3
 ...
(1 - rate_of_each_table) ^ (table_count - 1)*rate_of_each_table*table_count
(1 - rate_of_each_table) ^ table_count * table_count

function ret = calc_mean_cryptanalysis_time(N,t,m,table_count,step_speed)
rate_of_each_table = calc_success_probability(N, t, m);
temp = rate_of_each_table;
all  = 0;
for i = 1 : table_count
    all = all + temp * i;
    temp = temp * (1 - rate_of_each_table);
end;
all = all + (1 - rate_of_each_table) ^ table_count * table_count;

ret = t*t/2 / step_speed * all;
\end{lstlisting}
\section{最大磁盘读取时间}
\begin{lstlisting}
function ret = calc_max_disk_access_time(m, table_count, disk_speed)
ret = m*16/1024/1024 * disk_speed * table_count;
\end{lstlisting}
\section{实际磁盘读取时间}
\begin{lstlisting}
                               rate_of_each_table * 1 
(1 - rate_of_each_table) ^ 1 * rate_of_each_table * 2 
(1 - rate_of_each_table) ^ 2 * rate_of_each_table * 3 
 ...
(1-rate_of_each_table)^(table_count-1)*rate_of_each_table*table_count
(1 - rate_of_each_table) ^ table_count * table_count 

function ret = calc_mean_disk_access_time(N,t,m,table_count,disk_speed)
rate_of_each_table = calc_success_probability(N, t, m);
temp = rate_of_each_table;
all  = 0;
for i = 1 : table_count
    all = all + temp * i;
    temp = temp * (1 - rate_of_each_table);
end;
all = all + (1 - rate_of_each_table) ^ table_count * table_count;

ret = m*16/1024/1024 * disk_speed * all;
\end{lstlisting}
\chapter{密钥字符集对照表}
\label{cha:a_c}
numeric            = [0123456789]\\
alpha              = [ABCDEFGHIJKLMNOPQRSTUVWXYZ]\\
alpha-numeric      = [ABCDEFGHIJKLMNOPQRSTUVWXYZ0123456789]\\
loweralpha         = [abcdefghijklmnopqrstuvwxyz]\\
loweralpha-numeric = [abcdefghijklmnopqrstuvwxyz0123456789]\\
mixalpha           = [abcdefghijklmnopqrstuvwxyzABCDEFGHIJKLMNOPQRSTUVWXYZ]\\
mixalpha-numeric   = [abcdefghijklmnopqrstuvwxyzABCDEFGHIJKLMNOPQRSTUVWXYZ0123456789]
