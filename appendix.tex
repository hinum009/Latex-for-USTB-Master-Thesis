\appendix
\chapter{MATLAB程序}
\section{破解成功率}
\begin{lstlisting}
\% 1 - (1 - 1 / N)^(m(1) + m(2) + m(3) + ... + m(t - 1))
\% m(1) = m, m(i) = N * (1 - (1 - 1 / N) ^ m(i - 1))
function ret = calc_success_probability(n, min, max, t, m, table_count)
format long g;
arr = zeros(1, t - 1);
arr(1) = m;
format long g;
N = calc_N(n,min,max);
for i = 2 : t - 1
    format long g;
	arr(i) = N * (1 - (1 - 1 / N) ^ arr(i - 1));
end;
exp = 0;
for i = 1 : t - 1
    format long g;
	exp = exp + arr(i);
end;
format long g;
success_probability_1 = 1 - (1 - 1 / N) ^ exp;
success_probability_table_count=1-(1-success_probability_1)^table_count;
ret = success_probability_table_count;

function ret = calc_N( n, min, max)
arr = zeros(1, max - 1);
for i = 1 : max
	arr(i) = n^i;
end;
exp = 0;
for i = 1 : max
	exp = exp + arr(i);
end;
ret = exp;
\end{lstlisting}
\section{磁盘空间与成功率}
\begin{lstlisting}
function advantage_of_multiple_table(N, t, disk_usage_min, disk_usage_max)
grid_count = 100;
disk_usage = linspace(disk_usage_min, disk_usage_max, grid_count);
m = disk_usage * 1024 * 1024 / 16;
success_probability = zeros(1, grid_count);

for table_count = 1 : 5
  for i = 1 : grid_count
    success_probability(i)=calc_success_probability(N,t,m(i)/table_count);
    success_probability(i)=1-(1-success_probability(i))^table_count;
  end;
  if table_count ==1
    plot(disk_usage, success_probability,'-c*');
  end
  if table_count ==2
    plot(disk_usage,success_probability,'-x')
   end
  if table_count == 3
    plot(disk_usage,success_probability,'k-.')
  end
  if table_count == 4
    plot(disk_usage,success_probability,'-+')
  end
  if table_count == 5
    plot(disk_usage,success_probability,'y-o')
  end
  if N == 8353082582 & t == 2100 & table_count == 5
    plot(8000000*16*5/1024/1024, 0.999,'s','MarkerFaceColor','b',
                          'MarkerEdgeColor','b','MarkerSize',3);
    text(8000000*16*5/1024/1024, 0.999,'G','HorizontalAlignment','right',
                                           'VerticalAlignment','bottom');
  end;
  hold on;
end;
legend_int = zeros(5 - 1 + 1, 1);
for table_count = 1 : 5
	legend_int(table_count - 1 + 1, 1) = table_count;
end;
legend_str = int2str(legend_int);
legend(legend_str);
for i = 1 : grid_count
	success_probability(i) = 0.999;
end;
plot(disk_usage, success_probability, 'r');

for i = 1 : grid_count
	success_probability(i) = 1;
end;
plot(disk_usage, success_probability, 'r');

xlabel('磁盘空间(MB)');
ylabel('成功概率');
\end{lstlisting}
\section{密钥空间}
\begin{lstlisting}
function ret = calc_N( n, min, max)
arr = zeros(1, max );
for i = 1 : max
	arr(i) = n^i;
end;

exp = 0;
for i = 1 : max
	exp = exp + arr(i);
end;
format long g;
ret = exp;
\end{lstlisting}
\section{彩虹链长度}
\begin{lstlisting}
function ret = calc_t(N, success_probability, m, table_count)
success_probability_each=1-exp(log(1-success_probability)/table_count);
exp_min = log(1 - success_probability_each) / log(1 - 1 / N);

exp_all = 0;
t = 0;
next = m;

while exp_all < exp_min
	exp_all = exp_all + next;
	t = t + 1;
	next = N * (1 - (1 - 1 / N) ^ next);

	if t > 131072	% too large
		break;
	end;
end;

ret = t;
\end{lstlisting}
\section{磁盘占用空间}
\begin{lstlisting}
function ret = calc_disk_usage(m, table_count)
ret = ceil(m*16*table_count / 1024 / 1024 );
\end{lstlisting}
\section{最大破解时间}
\begin{lstlisting}
function ret = calc_max_cryptanalysis_time(t, table_count, step_speed)
ret = t*t/2 / step_speed * table_count;
\end{lstlisting}
\section{实际破解时间}
\begin{lstlisting}
                               rate_of_each_table * 1 
(1 - rate_of_each_table) ^ 1 * rate_of_each_table * 2 
(1 - rate_of_each_table) ^ 2 * rate_of_each_table * 3
 ...
(1 - rate_of_each_table) ^ (table_count - 1)*rate_of_each_table*table_count
(1 - rate_of_each_table) ^ table_count * table_count

function ret = calc_mean_cryptanalysis_time(N,t,m,table_count,step_speed)
rate_of_each_table = calc_success_probability(N, t, m);
temp = rate_of_each_table;
all  = 0;
for i = 1 : table_count
    all = all + temp * i;
    temp = temp * (1 - rate_of_each_table);
end;
all = all + (1 - rate_of_each_table) ^ table_count * table_count;

ret = t*t/2 / step_speed * all;
\end{lstlisting}
\section{最大磁盘读取时间}
\begin{lstlisting}
function ret = calc_max_disk_access_time(m, table_count, disk_speed)
ret = m*16/1024/1024 * disk_speed * table_count;
\end{lstlisting}
\section{实际磁盘读取时间}
\begin{lstlisting}
                               rate_of_each_table * 1 
(1 - rate_of_each_table) ^ 1 * rate_of_each_table * 2 
(1 - rate_of_each_table) ^ 2 * rate_of_each_table * 3 
 ...
(1-rate_of_each_table)^(table_count-1)*rate_of_each_table*table_count
(1 - rate_of_each_table) ^ table_count * table_count 

function ret = calc_mean_disk_access_time(N,t,m,table_count,disk_speed)
rate_of_each_table = calc_success_probability(N, t, m);
temp = rate_of_each_table;
all  = 0;
for i = 1 : table_count
    all = all + temp * i;
    temp = temp * (1 - rate_of_each_table);
end;
all = all + (1 - rate_of_each_table) ^ table_count * table_count;

ret = m*16/1024/1024 * disk_speed * all;
\end{lstlisting}
\chapter{彩虹表部分实现代码}
\section{彩虹表生成}
\begin{lstlisting}
#ifdef _WIN32
	#pragma warning(disable : 4786)
#endif
#ifdef _WIN32
	#include <windows.h>
#else
	#include <unistd.h>
#endif
#include <time.h>
#include "ChainWalkContext.h"

void Usage()
{
	Logo();

	printf("usage: rtgen hash_algorithm \\\n");
	printf("       plain_charset plain_len_min plain_len_max \\\n");
	printf("       rainbow_table_index \\\n");
	printf("       rainbow_chain_length rainbow_chain_count \\\n");
	printf("       file_title_suffix\n");
	printf("       rtgen hash_algorithm \\\n");
	printf("       plain_charset plain_len_min plain_len_max \\\n");
	printf("       rainbow_table_index \\\n");
	printf("             -bench\n");
	printf("\n");

	CHashRoutine hr;
	printf("plain_charset: use any charset name in charset.txt here\n");
	printf("plain_len_min:        min length of the plaintext\n");
	printf("plain_len_max:        max length of the plaintext\n");
	printf("rainbow_table_index:  index of the rainbow table\n");
	printf("rainbow_chain_length: length of the rainbow chain\n");
	printf("rainbow_chain_count\n");
	printf("file_title_suffix\n");
	printf("add your comment of the generated rainbow table here\n");
	printf("-bench:               do some benchmark\n");

	printf("\n");
	printf("example: rtgen lm alpha 1 7 0 100 16 test\n");
	printf("         rtgen md5 byte 4 4 0 100 16 test\n");
	printf("         rtgen sha1 numeric 1 10 0 100 16 test\n");
	printf("         rtgen lm alpha 1 7 0 -bench\n");
}

void Bench(string sHashRoutineName, string sCharsetName, int nPlainLenMin, 
                                int nPlainLenMax, int nRainbowTableIndex)
{
  // Setup CChainWalkContext
  if (!CChainWalkContext::SetHashRoutine(sHashRoutineName))
  {
    printf("hash routine %s not supported\n", sHashRoutineName.c_str());
    return;
  }
  if (!CChainWalkContext::SetPlainCharset(sCharsetName, 
                           nPlainLenMin, nPlainLenMax))
    return;
  if (!CChainWalkContext::SetRainbowTableIndex(nRainbowTableIndex))
  {
    printf("invalid rainbow table index %d\n", nRainbowTableIndex);
    return;
  }

  // Bench hash
  {
	CChainWalkContext cwc;
	cwc.GenerateRandomIndex();
	cwc.IndexToPlain();

	clock_t t1 = clock();
	int nLoop = 2500000;
	int i;
	for (i = 0; i < nLoop; i++)
		cwc.PlainToHash();
	clock_t t2 = clock();
	float fTime = 1.0f * (t2 - t1) / CLOCKS_PER_SEC;
	}

	// Bench step
	{
	CChainWalkContext cwc;
	cwc.GenerateRandomIndex();

	clock_t t1 = clock();
	int nLoop = 2500000;
	int i;
	for (i = 0; i < nLoop; i++)
	{
		cwc.IndexToPlain();
		cwc.PlainToHash();
		cwc.HashToIndex(i);
	}
	clock_t t2 = clock();
	float fTime = 1.0f * (t2 - t1) / CLOCKS_PER_SEC;

	}
}

int main(int argc, char* argv[])
{
  if (argc == 7)
  {
    if (strcmp(argv[6], "-bench") == 0)
    {
	Bench(argv[1],argv[2],atoi(argv[3]),atoi(argv[4]),atoi(argv[5]));
	return 0;
    }
  }

  if (argc != 9)
  {
    Usage();
    return 0;
  }

  string sHashRoutineName  = argv[1];
  string sCharsetName      = argv[2];
  int nPlainLenMin         = atoi(argv[3]);
  int nPlainLenMax         = atoi(argv[4]);
  int nRainbowTableIndex   = atoi(argv[5]);

  int nRainbowChainLen     = atoi(argv[6]);
  int nRainbowChainCount   = atoi(argv[7]);
  string sFileTitleSuffix  = argv[8];

  // nRainbowChainCount check
  if (nRainbowChainCount >= 134217728)
  {
    printf("this will not support generate a table larger than 2GB\n");
    printf("please use a smaller rainbow_chain_count\n");
    return 0;
  }

  // Setup CChainWalkContext
  if (!CChainWalkContext::SetHashRoutine(sHashRoutineName))
  {
    printf("hash routine %s not supported\n", sHashRoutineName.c_str());
    return 0;
  }
  if (!CChainWalkContext::SetPlainCharset(sCharsetName, nPlainLenMin,
                                                      nPlainLenMax))
	return 0;
  if (!CChainWalkContext::SetRainbowTableIndex(nRainbowTableIndex))
  {
	printf("invalid rainbow table index %d\n", nRainbowTableIndex);
	return 0;
  }
  CChainWalkContext::Dump();

// Low priority
#ifdef _WIN32
	SetThreadPriority(GetCurrentThread(), THREAD_PRIORITY_IDLE);
#else
	nice(19);
#endif

// FileName
char szFileName[256];
sprintf(szFileName,"%s_%s#%d-%d_%d_%dx%d_%s.rt", sHashRoutineName.c_str(),
	 sCharsetName.c_str(),
	 nPlainLenMin,
	 nPlainLenMax,
	 nRainbowTableIndex,
	 nRainbowChainLen,
	 nRainbowChainCount,
	 sFileTitleSuffix.c_str());

// Open file
	fclose(fopen(szFileName, "a"));
	FILE* file = fopen(szFileName, "r+b");
	if (file == NULL)
	{
		printf("failed to create %s\n", szFileName);
		return 0;
	}

// Check existing chains
	unsigned int nDataLen = GetFileLen(file);
	nDataLen = nDataLen / 16 * 16;
	if (nDataLen == nRainbowChainCount * 16)
	{
	  printf("precomputation of rainbow tables already finished\n");
	  fclose(file);
	  return 0;
	}
	if (nDataLen > 0)
		printf("continuing from interrupted precomputation...\n");
	fseek(file, nDataLen, SEEK_SET);

// Generate rainbow table
printf("generating...\n");
CChainWalkContext cwc;
clock_t t1 = clock();
int i;
for (i = nDataLen / 16; i < nRainbowChainCount; i++)
{
	cwc.GenerateRandomIndex();
	uint64 nIndex = cwc.GetIndex();
	if (fwrite(&nIndex, 1, 8, file) != 8)
	{
		printf("disk write fail\n");
		break;
	}

	int nPos;
	for (nPos = 0; nPos < nRainbowChainLen - 1; nPos++)
	{
		cwc.IndexToPlain();
		cwc.PlainToHash();
		cwc.HashToIndex(nPos);
	}

	nIndex = cwc.GetIndex();
	if (fwrite(&nIndex, 1, 8, file) != 8)
	{
		printf("disk write fail\n");
		break;
	}

	if ((i + 1) % 100000 == 0 || i + 1 == nRainbowChainCount)
	{
		clock_t t2 = clock();
		int nSecond = (t2 - t1) / CLOCKS_PER_SEC;
		printf("%d of %d chains generated (%d m %d s)\n",i+1,
			 nRainbowChainCount,
			 nSecond / 60,
			 nSecond % 60);
			 t1 = clock();
	}
}
// Close
fclose(file);
return 0;
}
\end{lstlisting}
\section{彩虹表破解}
\begin{lstlisting}
void GetTableList(string sWildCharPathName, vector<string>& vPathName)
{
  vPathName.clear();

  string sPath;
  int n = sWildCharPathName.find_last_of('\\');
  if (n != -1)
	sPath = sWildCharPathName.substr(0, n + 1);

  _finddata_t fd;
  long handle = _findfirst(sWildCharPathName.c_str(), &fd);
  if (handle != -1)
  {
    do
    {
	string sName = fd.name;
	if (sName != "." && sName != ".." && !(fd.attrib & _A_SUBDIR))
	{
		string sPathName = sPath + sName;
		vPathName.push_back(sPathName);
	}
    } while (_findnext(handle, &fd) == 0);

    _findclose(handle);
  }
}
#else
void GetTableList(int argc, char* argv[], vector<string>& vPathName)
{
	vPathName.clear();

	int i;
	for (i = 1; i <= argc - 3; i++)
	{
		string sPathName = argv[i];

		struct stat buf;
		if (lstat(sPathName.c_str(), &buf) == 0)
		{
			if (S_ISREG(buf.st_mode))
				vPathName.push_back(sPathName);
		}
	}
}
#endif

bool NormalizeHash(string& sHash)
{
	string sNormalizedHash = sHash;

	if (   sNormalizedHash.size() % 2 != 0
		|| sNormalizedHash.size() < MIN_HASH_LEN * 2
		|| sNormalizedHash.size() > MAX_HASH_LEN * 2)
		return false;

// Make lower
int i;
for (i = 0; i < sNormalizedHash.size(); i++)
{
  if (sNormalizedHash[i] >= 'A' && sNormalizedHash[i] <= 'F')
  sNormalizedHash[i] = sNormalizedHash[i] - 'A' + 'a';
}

// Character check
for (i = 0; i < sNormalizedHash.size(); i++)
{
  if ((sNormalizedHash[i] < 'a' || sNormalizedHash[i] > 'f')
	&& (sNormalizedHash[i] < '0' || sNormalizedHash[i] > '9'))
	return false;
}

sHash = sNormalizedHash;
return true;
}

void LoadLMHashFromPwdumpFile(string sPathName, vector<string>& vUserName, 
                       vector<string>& vLMHash, vector<string>& vNTLMHash)
{
  vector<string> vLine;
  if (ReadLinesFromFile(sPathName, vLine))
  {
    int i;
    for (i = 0; i < vLine.size(); i++)
    {
      vector<string> vPart;
      if (SeperateString(vLine[i], "::::", vPart))
      {
      string sUserName = vPart[0];
      string sLMHash   = vPart[2];
      string sNTLMHash = vPart[3];

      if (sLMHash.size() == 32 && sNTLMHash.size() == 32)
      {
	if (NormalizeHash(sLMHash) && NormalizeHash(sNTLMHash))
	{
	  vUserName.push_back(sUserName);
	  vLMHash.push_back(sLMHash);
	  vNTLMHash.push_back(sNTLMHash);
	}
       }
    }
  }
}
else
  printf("can't open %s\n", sPathName.c_str());
}

bool NTLMPasswordSeek(unsigned char* pLMPassword, int nLMPasswordLen,
    int nLMPasswordNext, unsigned char* pNTLMHash, string& sNTLMPassword)
{
  if (nLMPasswordNext == nLMPasswordLen)
  {
    unsigned char md[16];
    MD4(pLMPassword, nLMPasswordLen * 2, md);
    if (memcmp(md, pNTLMHash, 16) == 0)
    {
      sNTLMPassword = "";
      int i;
      for (i = 0; i < nLMPasswordLen; i++)
      sNTLMPassword += char(pLMPassword[i * 2]);
      return true;
     }
    else
      return false;
    }

if (NTLMPasswordSeek(pLMPassword,nLMPasswordLen,nLMPasswordNext+1,
                                        pNTLMHash, sNTLMPassword))
  return true;

if (   pLMPassword[nLMPasswordNext * 2] >= 'A'
	&& pLMPassword[nLMPasswordNext * 2] <= 'Z')
{
  pLMPassword[nLMPasswordNext * 2] = pLMPassword[nLMPasswordNext * 2]
                                                      - 'A' + 'a';
  if (NTLMPasswordSeek(pLMPassword,nLMPasswordLen,nLMPasswordNext+1,
                                          pNTLMHash, sNTLMPassword))
    return true;
  pLMPassword[nLMPasswordNext * 2] = pLMPassword[nLMPasswordNext * 2]
   							 - 'a' + 'A';
}

return false;
}

bool LMPasswordCorrectCase(string sLMPassword,
	            unsigned char* pNTLMHash, string& sNTLMPassword)
{
  if (sLMPassword.size() == 0)
  {
    sNTLMPassword = "";
    return true;
  }

unsigned char* pLMPassword = new unsigned char[sLMPassword.size() * 2];
int i;
for (i = 0; i < sLMPassword.size(); i++)
{
  pLMPassword[i * 2    ] = sLMPassword[i];
  pLMPassword[i * 2 + 1] = 0x00;
}
bool fRet = NTLMPasswordSeek(pLMPassword, sLMPassword.size(), 0,
                                           pNTLMHash, sNTLMPassword);
delete pLMPassword;

return fRet;
}

void Usage()
{
  Logo();

  printf("usage: rcrack rainbow_table_pathname -h hash\n");
  printf("       rcrack rainbow_table_pathname -l hash_list_file\n");
  printf("       rcrack rainbow_table_pathname -f pwdump_file\n");
  printf("rainbow_table_pathname: pathname of the rainbow table(s),
					 wildchar(*, ?) supported\n");
  printf("-h hash:                use raw hash as input\n");
  printf("-l hash_list_file:      use hash list file as input,
 					 each hash in a line\n");
  printf("-f pwdump_file: use pwdump file as input, this will handle 
					    lanmanager hash only\n");
  printf("\n");
  printf("example: rcrack *.rt -h 5d41402abc4b2a76b9719d911017c592\n");
  printf("         rcrack *.rt -l hash.txt\n");
  printf("         rcrack *.rt -f hash.txt\n");
}

int main(int argc, char* argv[])
{
  if (argc < 4)
  {
    Usage();
    return 0;
   }
   string sInputType        = argv[argc - 2];
   string sInput            = argv[argc - 1];

// vPathName
vector<string> vPathName;
GetTableList(argc, argv, vPathName);
if (vPathName.size() == 0)
{
  printf("no rainbow table found\n");
    return 0;
}

// fCrackerType, vHash, vUserName, vLMHash
bool fCrackerType;		// true: hash cracker, false: lm cracker
vector<string> vHash;		// hash cracker
vector<string> vUserName;	// lm cracker
vector<string> vLMHash;		// lm cracker
vector<string> vNTLMHash;	// lm cracker
if (sInputType == "-h")
{
  fCrackerType = true;

  string sHash = sInput;
  if (NormalizeHash(sHash))
    vHash.push_back(sHash);
  else
    printf("invalid hash: %s\n", sHash.c_str());
}
else if (sInputType == "-l")
{
  fCrackerType = true;
  string sPathName = sInput;
  vector<string> vLine;
  if (ReadLinesFromFile(sPathName, vLine))
  {
    int i;
    for (i = 0; i < vLine.size(); i++)
    {
      string sHash = vLine[i];
      if (NormalizeHash(sHash))
        vHash.push_back(sHash);
      else
	printf("invalid hash: %s\n", sHash.c_str());
    }
  }
  else
    printf("can't open %s\n", sPathName.c_str());
}
else if (sInputType == "-f")
{
  fCrackerType = false;
  string sPathName = sInput;
  LoadLMHashFromPwdumpFile(sPathName, vUserName, vLMHash, vNTLMHash);
}
else
{
  Usage();
  return 0;
}	
if (fCrackerType && vHash.size() == 0)
 	return 0;
if (!fCrackerType && vLMHash.size() == 0)
 	return 0;

// hs
CHashSet hs;
if (fCrackerType)
{
  int i;
  for (i = 0; i < vHash.size(); i++)
  hs.AddHash(vHash[i]);
}
else
{
  int i;
  for (i = 0; i < vLMHash.size(); i++)
  {
  hs.AddHash(vLMHash[i].substr(0, 16));
  hs.AddHash(vLMHash[i].substr(16, 16));
  }
}

// Run
CCrackEngine ce;
ce.Run(vPathName, hs);
}
\end{lstlisting}
