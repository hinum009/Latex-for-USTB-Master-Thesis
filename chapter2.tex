\chapter{密码分析学}
\section{密码分析学的发展背景}
{ 静态测试技术是指不运行被测程序本身,而是通过分析源程序的语法、结构、过程、接口等来检查程序的正确性。主要包括源代码审核和二进制审核[5] 。

源代码审核是指通过人工或相关工具对源代码进行审核,找出其中存在的漏洞。源代码分析是一项非常重要的措施,因为它可以在软件发布之前对潜在的漏洞进行发掘,提前找到可能的漏洞,以减少软件在实际运行过程中因漏洞而导致的损失。目前,世界各个大型的软件公司基本都有专门的小组负责对自身的软件进行安全测试,他们所使用的主要技术之一就是源代码审核技术。2006年,美国政府与专业的源码审核公司Coverity达成了协议,由政府拨专款资助Coverity公司分析源代码并解决美国各级基础设施的关键代码的安全隐患。

二进制审核是指对二进制代码进行反汇编,通过分析反汇编代码来寻找安全漏洞。通常这需要借助于反汇编工具将待检测的可执行文件进行反汇编,将难以被辨认的机器码解析为某种更适合人理解的汇编代码,同时也还会用到其它一些反编译器和调试器等来分析和追踪寄存器值,从而来发现可能存在的漏洞。从这里也可以看出,二进制审核由于没有源代码,所以对漏洞检查的安全人员有着更高的要求。

目前,静态分析技术的目标一般是C和C++,而且由于C++的一些高级特性使得源代码审核更加困难,所以大部分的静态分析工具集中在C语言上。对于Java语言,目前也有一些审核工具,如Findbugs[7] ,另外,国内著名安全组织安全焦点在2007会议上也有文献讨论基于Java的Web应用程序的安全问题。

静态代码分析技术包含比较多的部分,如词法分析、控制流分析、数据流分析、符号执行、类型推理、抽象解释、模型检测和自动定理证明等等。目前在安全漏洞挖掘方向使用最为广泛的是词法分析、控制流分析和数据流分析[6] 。此外,国外的研究人员在2007年度新提出了应用布尔可满足概念来改进源码审核技术的有效性,这也是当前源码审核技术研究的一个焦点。

但是静态测试技术也有着它的缺点,首先是源代码审核的方法要求提供源代码,但是一些涉及到重要内容的Web应用往往不会提供源代码,有些公司提供的都是编译好的可执行程序,因此源代码审核并不适用于大部分的Web应用的安全测试。其次,由于静态测试时针对静态的程序为目标,难以发现程序在动态运行过程中存在的问题。

}
\section{密码分析学的定义}
\section{密码分析学研究的必要性}
\section{密码分析的方法}
\subsection{穷举法}
\subsection{查表法}
\subsection{时空折中法}
{发酵法阿斯顿发阿斯顿发阿斯顿发阿斯顿发阿斯顿发

法士大夫阿斯顿发阿德说发

阿斯顿发生的f}
\subsection{分析法}
